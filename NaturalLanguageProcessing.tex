\chapter{Natural Language Processing}\label{ch_nlp}
\chapterauthor{Jeff Yoshimi}

\textit{Some introduction.}

Some very simple approaches used historically:
- Manual feature vectors.  Example: the 6 bit vectors in Elman's "Finding structure in time"
- One-hot codings as a very simple approach. Example: the 31 bit vectors  in Elman's "Finding structure in time"

Bag of words
LSA

To begin to get a feel for some of this see

http://vectors.nlpl.eu/explore/embeddings/en/

\section{Distributional Semantics Theory}
Harris, Firth, Wittgenstein.
Know a word by the company it keeps.
Few examples illustrating it.

\textit{Brunila \& LaViolette's paper revisiting the idea of distributional semantics: \url{https://arxiv.org/pdf/2205.07750.pdf} (URL for now \dots)}

\section{Word Embeddings and Co-occurrence Matrices}

Preprocessing, normalization.

More basic NLP, though may have overlap with data science chapter. 

See Lenci review paper.
Walkthrough of the different steps: tokenization, co-occurrences (window size, skip-gram, stop-word filtering), PPMI transform, etc.
\textit{Should we reference the functions from SimBrain?}

Point about corpus quality.

\subsection{Evaluation}
Different gold standards of evaluation (SimLex, SimVerb, etc.).
Similarity vs assocaition vs relatedness.

\subsection{Different spins}
Paragram embeddings.
Counterfitting / retrofitted embeddings.
Multilingual embeddings.
Different associative models (e.g., Mike Jones' BEAGLE).
Co-occurrence vs next-word prediction.

\section{Applications}
Word similarity -> cosine similarity.
Discourse tracking.
Machine translation.
Sentiment analysis.
Language change (HistWords).

\section{Shortcomings}
Polysemy and context.
Dimensionality of representations.

\section{Theoretical implications?}
Points from Boleda paper: semantic change, polysemy and composition, and the grammar-semantics interface.
\textit{Separate from shortcomings or combine these two?}

\section{Trajectory of the field}
Improvements over the years (skip-gram, counterfitting, etc.).
Transition to Language Models?

\section{Exercises}

\textit{Should have some brief intro before the questions\dots}

\subsection{Basic NLP}

\begin{enumerate}
\item Walkthrough a simple example of counting co-occurrences.
\item How does window size impact the embeddings?
\item What is potential motivation for using the \textit{skip-gram} setting?
\item Calculate the similarity between these words: [list of words]. 
\item Something about the PPMI transformation?
\end{enumerate}

\subsection{Geometric thinking}

\begin{enumerate}
\item What does it mean for a word embedding to be n-dimensional?
\item Why is cosine similarity used over other distance functions, like traditional euclidean distance?
\item Something where they perform clustering?
\end{enumerate}

\subsection{Corpus quality}

\begin{enumerate}
\item With the limited (demo) training corpus, how well do you think it captures the actual meaning of the words?
\item Does corpus size or quality matter? (Too basic of a question.)
\item Something with a corpus that deals with polysemy (financial institution \& geographic texts)
\end{enumerate}

\subsection{Neural Networks and other advances}

\begin{enumerate}
\item SRNs?
\item Next-word prediction?
\item Text generation?
\end{enumerate}